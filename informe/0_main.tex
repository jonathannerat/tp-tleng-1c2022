% !TeX spellcheck = es_AR
\documentclass[10pt, a4paper]{article}

\usepackage{caratula}

\usepackage[utf8]{inputenc}
\usepackage[spanish]{babel} % separación silábica en castellano

%\usepackage[paper=a4paper, margin=2cm]{geometry} % especifico márgenes manualmente
\usepackage{a4wide} % margenes un poco más anchos

%\usepackage{caratula/caratula}
%\graphicspath{ {images/} }
%\usepackage{tikz}
%\usepackage{tikz-qtree}
%\usepackage{framed}
%\usepackage{array}
%\usepackage{tabular}

\usepackage{microtype} % saca warnings de underfull boxes
\usepackage{fancyhdr} % encabezado y pie de página
\usepackage{lastpage} % para que muestre la última página en footer

% Comandos para figuras, graficos, tikz, etc.
\usepackage{tikz}
\usepackage{epsfig}
\usepackage{graphicx}
\usepackage{epsfig}
\usepackage{caption}
\usepackage{subcaption}
\usepackage{svg}
\usepackage{listings}

% Estilos para lstlisting
\definecolor{codegreen}{rgb}{0,0.6,0}
\definecolor{codegray}{rgb}{0.5,0.5,0.5}
\definecolor{codepurple}{rgb}{0.58,0,0.82}
\definecolor{backcolour}{rgb}{0.97,0.97,0.94}

\lstdefinestyle{mystyle}{
    keywordstyle=\color{magenta},
    stringstyle=\color{codepurple},
    basicstyle=\ttfamily\footnotesize,
    breakatwhitespace=false,
    breaklines=true,
    captionpos=b,
    keepspaces=true,
    showspaces=false,
    showstringspaces=false,
    showtabs=false,
    tabsize=2
}

\lstset{style=mystyle}
\lstset{
     literate=%
         {á}{{\'a}}1
         {í}{{\'i}}1
         {é}{{\'e}}1
         {ú}{{\'u}}1
         {ó}{{\'o}}1
}

% Comandos para teoremas, etc.
\usepackage{amsmath}
\usepackage{amsthm}
\usepackage{amssymb}

% comandos para algoritmos
\usepackage{algorithm}
\usepackage[noend]{algpseudocode}
\algnewcommand{\IfThenElse}[3]{% \IfThenElse{<if>}{<then>}{<else>}
\State \algorithmicif\ #1\ \algorithmicthen\ #2\ \algorithmicelse\ #3}
\algnewcommand{\IfThen}[2]{% \IfThenElse{<if>}{<then>}
  \State \algorithmicif\ #1\ \algorithmicthen\ #2}
  
\newcommand{\norm}[1]{\left\lVert#1\right\rVert_2}
\usepackage{hyperref}
\hypersetup{
    colorlinks=true,
    linkcolor=blue,
    filecolor=magenta,      
    urlcolor=cyan,
    pdftitle={TP TLENG 1c2022},
    pdfpagemode=UseNone,
    citecolor=blue,
}

\pagestyle{fancy} % Acomodo el encabezado y pie de página.
\lhead{Teoría de lenguajes}
\setlength{\headheight}{12pt}
\rhead{Primer Cuatrimestre 2022}
\cfoot{\thepage /\pageref{LastPage}}



\begin{document}

\titulo{Trabajo Práctico}
\subtitulo{Generador de JSON de ejemplo}

\fecha{\today}

\materia{Teoría de Lenguajes}

\integrante{Franco Demarco}{348/19}{franco.demarco3400@gmail.com}
\integrante{Damian}{Y/Y}{damian@example.com}
\integrante{Jonathan Teran Carballo}{643/18}{jonathan.nerat@gmail.com}



\maketitle

\tableofcontents

\newpage

\section{Resumen}

En este trabajo implementaremos un programa para parsear definiciones de un subconjunto de tipos de Go, y generar cadenas JSON aleatorias que se correspondan a los tipos definidos.

El trabajo estará organizado de la siguiente manera:

\begin{itemize}
	\item \textbf{Sección \ref{sec:intro}: Introducción}, requerimientos del programa, y las herramientas que utilizamos para implementarlo
	\item \textbf{Sección \ref{sec:gramatica}: Gramática, Parser y AST}, gramática utilizada, implementación de la misma y el árbol sintáctico resultante.
	\item \textbf{Sección \ref{sec:referencias}: Ciclos y Referencias} métodos utilizados para detectar ciclos y resolver referencias entre tipos
	\item \textbf{Sección \ref{sec:json}: Generación de JSON} métodos utilizados para generar la salida aleatoria en formato JSON
\end{itemize}

Palabras clave: \textit{Parser, LALR(1), JSON Generator}

\newpage

\section{Introducción}
\label{sec:intro}

El programa deberá ser capaz de leer definiciones de un subconjunto de tipos de Go. Estas definiciones consisten de asignarle un nombre a un tipo, donde el tipo puede ser:

\begin{itemize}
    \item Tipo básico: \texttt{string}, \texttt{int}, \texttt{float64} y \texttt{bool}. Representan los valores convencionales.
    \item Arreglos: tienen la forma \texttt{[]}\textit{T} y representan listas de elementos de tipo \texttt{T}. Por ejemplo: \texttt{[]float64}, \texttt{[][]int}, etc. 
    \item Estructuras: tienen la forma \texttt{struct \{ prop } \textit{T} \texttt{ ... \}}, donde prop es el nombre de una propiedad de la estructura y T el tipo de la misma. Las estructuras tienen más de una propiedad, cada una con un tipo asociado. Por ejemplo:
    \begin{verbatim}
struct {
    nombre string
    pais struct {
        nombre string
        poblacion int
    }
}
    \end{verbatim}
    \item Referencias: son nombres que hacen referencia a otro tipo definido en la entrada. Por ejemplo:
    \begin{verbatim}
type persona {
    nombre cadena
    edad entero
}
type cadena string
type entero int
    \end{verbatim}
\end{itemize}

Para hacer el parsing de estas estructuras, utilizaremos la librería PLY (\textit{Python Lex-Yacc}). Esta consiste de dos módulos principales: Lex y Yacc. Lex es el que se encargará de leer la entrada y convertirla en una lista de lexemas, mientras que Yacc nos permitirá definir una gramática (utilizando estos lexemas como terminales) y generar el parser LALR(1) de la misma.


\section{Gramática, Parser y AST}
\label{sec:gramatica}

\subsection{Construcción de la Gramática}

Para construir la gramática, analizaremos una posible cadena de entrada para
ver cuáles son los terminales y no terminales necesarios:

\begin{verbatim}
type persona struct {
    nombre string
    edad int
    nacionalidad pais
    ventas []float64
    activo bool
}
type pais struct {
    nombre string
    codigo struct {
        prefijo string
        sufijo string
    }
}
\end{verbatim}

Como en la entrada podemos recibir muchas declaraciones de tipos, y por lo
menos hay una declaración (que es la que se utiliza para generar el JSON de
salida), intuimos que debe haber una producción para la declaración, y otra
para acumular varias:

\begin{equation}
\begin{split}
DeclList & \rightarrow Decl~|~Decl~DeclList 
\end{split}
\end{equation}

Cada declaración consiste de un nombre seguido del tipo asociado al nombre:

\begin{equation} \label{gram:decl}
\begin{split}
Decl & \rightarrow type~name~Type
\end{split}
\end{equation}

El no terminal \texttt{Type} debe ser capaz de generar los tipos mencionados en
la sección \ref{sec:intro}. 

\begin{equation} \label{gram:type}
\begin{split}
Type & \rightarrow BasicType~|~ArrayType~|~StructType \\
BasicType & \rightarrow string~|~int~|~float64~|~bool \\
ArrayType & \rightarrow []~Type
\end{split}
\end{equation}

Para el no terminal \texttt{StructType}, recordemos que las estructuras pueden
contener más de una propiedad, por lo que necesitaremos de otro no terminal
para listar las mismas:

\begin{equation} \label{gram:struct}
\begin{split}
StructType & \rightarrow struct~\{~PropList~\} \\
PropList & \rightarrow Prop~|~Prop~PropList \\
Prop & \rightarrow name~Type
\end{split}
\end{equation}

Para simplificar la gramática, podemos juntar los terminales \texttt{name},
\texttt{string}, \texttt{int}, \texttt{float64} y \texttt{bool} en un terminal
\texttt{id}, que representa los identificadores tanto de propiedades como de
tipos. Luego en la etapa de parsing los diferenciaremos según el contexto y el
valor que contenga el lexema. Reemplazando por \texttt{id} en \ref{gram:decl},
\ref{gram:type} y \ref{gram:struct}, y juntando todo, obtenemos el siguiente
conjunto de producciones $P'$:


\begin{equation}
\begin{split}
DeclList & \rightarrow Decl~|~Decl~DeclList  \\
Decl & \rightarrow type~id~Type \\
Type & \rightarrow BasicType~|~ArrayType~|~StructType \\
BasicType & \rightarrow id \\
ArrayType & \rightarrow []~Type \\
StructType & \rightarrow struct~\{~PropList~\} \\
PropList & \rightarrow Prop~|~Prop~PropList \\
Prop & \rightarrow id~Type
\end{split}
\end{equation}

Podemos simplificarla un poco más si reemplazamos los no terminales del lado
derecho de \texttt{Type} con las producciones correspondientes, obteniendo así
$P$:

\begin{equation}
\begin{split}
DeclList & \rightarrow Decl~|~Decl~DeclList  \\
Decl & \rightarrow type~id~Type \\
Type & \rightarrow id~|~[]~Type~|~struct~\{~PropList~\} \\
PropList & \rightarrow Prop~|~Prop~PropList \\
Prop & \rightarrow id~Type
\end{split}
\end{equation}

Finalmente, proponemos la siguiente gramática

\[
G = \langle\{Decl, DeclList, Type, PropList, Prop\}, \{type, id, [], struct, '\{', '\}'\}, P, DeclList\rangle
\]

para generar el lenguaje de las cadenas que esperamos recibir por entrada.

\subsection{Implementación del Parser}

Antes de implementar la gramática con Yacc, tenemos que hacer un
preprocesamiento de la misma para convertir el texto en una lista de tokens.
Para esto utilizaremos el módulo Lex, y proveeremos las expresiones regulares
necesarias para que el lexer identifique a cada token.

Definimos 6 tokens en total, de los cuales 2 son los literales '\{' y '\}'. El
resto son:

\begin{itemize}
    \item \texttt{TYPE = type}
    \item \texttt{STRUCT = struct} 
    \item \texttt{ARRAY = []} 
    \item \texttt{ID = r'[a-z]\textbackslash w*'}
\end{itemize}

Los primeros 3 los definimos con la cadena que los identifica, mientras que el
token \texttt{ID} lo representamos con una expresion regular que corresponde
con los posibles identificadores de este lenguaje, palabras que empiecen con
una letra minúscula.

Con los tokens declarados, la implementación de la gramática es directa ya que
definimos un método por cada producción, o por cada no terminal del lado
izquierdo si sus producciones son similares. Además, en cada método se debe
armar progresivamente el arbol sintáctico (\textit{AST}), por lo que debemos
declarar algunas estructuras para facilitar su contrucción.

\subsection{AST}
\label{subsec:ast}

Para armar el arbol, asociaremos una estructura a cada no terminal:

\begin{itemize}
    \item \texttt{DeclList} y \texttt{PropList} se pueden implementar con los
        arreglos convencionales.
    \item \texttt{Decl}: estructura \texttt{TypedefNode} con las propiedades
        \textit{nombre (string)} y \textit{tipo (estructura de tipo)}.
    \item \texttt{Type}: son las estructuras de tipo, y dependen de la
        producción utilizada:
    \begin{itemize}
        \item \texttt{ID}: Si la cadena asociado al terminal es un tipo básico,
            lo representaremos con una estructura \texttt{BasicTypeNode} que
            contenga el nombre del tipo. Si no, lo representamos con la cadena
            original. Esta decisión será más evidente cuando hagamos la
            resolución de referencias en la seccion \ref{sec:referencias}.
        \item \texttt{[] Type}: estructura \texttt{ArrayTypeNode} con la
            propiedad \textit{tipo (estructura de tipo)}.
        \item \texttt{Struct \{ PropList \}}: estructura
            \texttt{StructTypeNode} con la propiedad \textit{props (lista de
            Prop)}
    \end{itemize}
    \item \texttt{Prop}: tupla $\langle \textit{nombre (string)}, \textit{tipo
        (estructura de tipo)}\rangle$
\end{itemize}

Luego, en el método de cada producción, instanciaremos la estructura asociada
al no terminal del lado izquierdo con las estructuras asociadas a los no
terminales del lado derecho. De esta forma, el resultado del parsing es el
árbol dado por el arreglo de estructuras \texttt{TypedefNode}: 

\begin{itemize}
    \item cada arreglo es un nodo que tiene como hijos a sus elementos,
    \item cada estructura es otro nodo, que tene como hijo a la estructura que
        se encuentra en su propiedad \textit{tipo},
    \item las hojas son los estructuras de tipos básicos, y las cadenas que no
        son tipos basicos (y por lo tanto, son referencias que se deben
        resolver).
\end{itemize}



\section{Ciclos y Referencias}
\label{sec:referencias}

Para completar el árbol y poder generar la salida, debemos antes resolver todas
las referencias que aparezcan en tipo principal (el primer elemento del arreglo
que devuelve nuestro parser). Esto no siempre será posible, ya que en algunos
casos estas referencias pueden generar ciclos.

\subsection{Detección de Ciclos}

Para detectar estos ciclos, armamos el grafo dirigido de dependencias asociado al tipo
principal, y lo recorremos marcando los nodos visitados hasta terminar de
recorrer todos (no hay ciclos), o hasta visitar un nodo previamente visitado
(hay un ciclo).

\begin{algorithm}[H]
\begin{algorithmic}
\Function{grafo\_dependencia}{$T$: TypedefNode[]}
    \State{$A \gets dict()$}
    \For {$t$ in $T$}
        \State {$A[t.nombre] \gets t.getDeps()$}
    \EndFor
    \State{\textbf{return} $A$}
\EndFunction
\end{algorithmic}
\caption{Construcción del grafo de dependencias}
\label{alg:grafo_dependencia}
\end{algorithm}

En el algoritmo \ref{alg:grafo_dependencia} llamamos al método \texttt{getDeps}
de TypedefNode, para obtener la lista de referencias. Este al mismo tiempo,
vuelve a llamar a getDeps para sus hijos en el arbol sintáctico, propagando las
cadenas encontrados nuevamente a la raíz.

\begin{algorithm}[H]
\begin{algorithmic}
    \Function{obtener\_ciclo}{$G$: grafo, $S$: string, $V$: diccionario}
    \State{\textbf{marcar} $S$ como visitado en $V$}
    \State{$siguientes \gets$ siguientes de $S$ en $G$}
    \For {$n$ in $siguientes$}
        \If{$n$ visitado en $V$}
            \State{\textbf{return}ref $\langle S, n \rangle$}
        \EndIf

        \State{$c \gets$ \texttt{obtener\_ciclo($G$, $S$, $V$)}}
        \If{$c$ no es None}
            \State{\textbf{return} $c$}
        \EndIf
    \EndFor
    \State{\textbf{return} None}
\EndFunction
\end{algorithmic}
\caption{Detección de ciclos}
\label{alg:obtener_ciclo}
\end{algorithm}

En el algoritmo \ref{alg:obtener_ciclo}, ademas de detectar si el grafo tiene un
ciclo o no, retornamos el arco del grafo que genera el ciclo para reportarlo
como error.

Notar que el grafo solamente tiene en cuenta las dependencias asociadas al tipo
principal, por lo que si en la entrada existen más definiciones de tipo que
generan dependencias circulares entre si, pero que no aparecen en el tipo
principal, el programa continuará sin reportar errores.

Por el contrario, si la detección de ciclos devuelve \texttt{None}, el tipo
principal no tiene dependencias circulares y podemos continuar con la resolución
de referencias.

\subsection{Resolución de referencias}

En la sección \ref{subsec:ast}, mencionamos que aquellos identificadores que no
sean tipos básicos, mantienen la cadena original. Para resolver estas
referencias, utilizaremos el resultado del parser (la lista de
\texttt{TypedefNode}'s) para crear un diccionario que tenga como claves los
las referencias, y como valor la estructura del tipo asociada.

Tanto la estructura \texttt{TypedefNode}, como las estructuras de tipos
tienen un método \texttt{resolve} que recibe como parámetro el diccionario
anterior. Cada uno de estos intercambian las referencias que aparecen en sus
propiedades por el valor correspondiente en el diccionario, y propagan la
llamada a \texttt{resolve} en sus propiedades.

Como verificamos que el tipo principal no tenía dependencias circulares, este
proceso de resolver cada nodo del árbol eventualmente termina, y tenemos un
arbol sintáctico completo, que podemos usar para generar la salida.



\section{Resolución y Generación de JSON}
\label{sec:json}


\section{Conclusiones}
\label{sec:conc}


\end{document}
