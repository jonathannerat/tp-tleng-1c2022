% !TeX spellcheck = es_AR
\documentclass[10pt, a4paper]{article}

\usepackage{caratula}

\usepackage[utf8]{inputenc}
\usepackage[spanish.lcroman]{babel}

%\usepackage[paper=a4paper, margin=2cm]{geometry} % especifico márgenes manualmente
\usepackage{a4wide} % margenes un poco más anchos

%\usepackage{caratula/caratula}
%\graphicspath{ {images/} }
%\usepackage{tikz}
%\usepackage{tikz-qtree}
%\usepackage{framed}
%\usepackage{array}
%\usepackage{tabular}

\usepackage{microtype} % saca warnings de underfull boxes
\usepackage{fancyhdr} % encabezado y pie de página
\usepackage{lastpage} % para que muestre la última página en footer

% Comandos para figuras, graficos, tikz, etc.
\usepackage{tikz}
\usepackage{epsfig}
\usepackage{graphicx}
\usepackage{epsfig}
\usepackage{caption}
\usepackage{subcaption}
\usepackage{svg}
\usepackage{listings}

% Estilos para lstlisting
\definecolor{codegreen}{rgb}{0,0.6,0}
\definecolor{codegray}{rgb}{0.5,0.5,0.5}
\definecolor{codepurple}{rgb}{0.58,0,0.82}
\definecolor{backcolour}{rgb}{0.97,0.97,0.94}

\lstdefinestyle{mystyle}{
    keywordstyle=\color{magenta},
    stringstyle=\color{codepurple},
    basicstyle=\ttfamily\footnotesize,
    breakatwhitespace=false,
    breaklines=true,
    captionpos=b,
    keepspaces=true,
    showspaces=false,
    showstringspaces=false,
    showtabs=false,
    tabsize=2
}

\lstset{style=mystyle}
\lstset{
     literate=%
         {á}{{\'a}}1
         {í}{{\'i}}1
         {é}{{\'e}}1
         {ú}{{\'u}}1
         {ó}{{\'o}}1
         {ñ}{{\~n}}1
}

% Comandos para teoremas, etc.
\usepackage{amsmath}
\usepackage{amsthm}
\usepackage{amssymb}

% comandos para algoritmos
\usepackage{algorithm}
\usepackage[noend]{algpseudocode}
\algnewcommand{\IfThenElse}[3]{% \IfThenElse{<if>}{<then>}{<else>}
\State \algorithmicif\ #1\ \algorithmicthen\ #2\ \algorithmicelse\ #3}
\algnewcommand{\IfThen}[2]{% \IfThenElse{<if>}{<then>}
  \State \algorithmicif\ #1\ \algorithmicthen\ #2}
  
\newcommand{\norm}[1]{\left\lVert#1\right\rVert_2}
\usepackage{biblatex}
\usepackage{csquotes}
\addbibresource{recursos.bib}
\usepackage{hyperref}
\hypersetup{
    colorlinks=true,
    linkcolor={blue!80!black},
    filecolor=magenta,      
    urlcolor=cyan,
    pdftitle={TP TLENG 1c2022},
    pdfpagemode=UseNone,
    citecolor=blue,
}

\pagestyle{fancy} % Acomodo el encabezado y pie de página.
\lhead{Teoría de lenguajes}
\setlength{\headheight}{12pt}
\rhead{Primer Cuatrimestre 2022}
\cfoot{\thepage /\pageref{LastPage}}
\setlength{\parskip}{0.5em}
\setlength{\parindent}{2em}

\renewcommand{\labelitemi}{\color{black}$\bullet$}
\renewcommand\labelitemii{\color{black}$\circ$}
\renewcommand\labelitemiii{\scriptsize \color{black}$\square$}
\renewcommand{\labelenumi}{\color{black}{\theenumi.}}

\begin{document}

\titulo{Trabajo Práctico}
\subtitulo{Generador de JSON de ejemplo}

\fecha{\today}

\materia{Teoría de Lenguajes}

\integrante{Franco Demarco}{348/19}{franco.demarco3400@gmail.com}
\integrante{Damián Fernando Huaier}{229/17}{damianhuaier@gmail.com}
\integrante{Jonathan Teran Carballo}{643/18}{jonathan.nerat@gmail.com}



\pagenumbering{roman} 
\maketitle
\phantomsection
\addcontentsline{toc}{section}{Índice general}
\tableofcontents
\phantomsection
\addcontentsline{toc}{section}{Resumen}


    En este trabajo implementaremos un programa para parsear definiciones de un subconjunto de tipos de Go \cite{Go}, y generar cadenas JSON \cite{JSON} aleatorias que se correspondan a los tipos definidos.

    El trabajo estará organizado de la siguiente manera:

\begin{itemize}
	\item \hyperlink{sec:intro}{Introducción}. Requerimientos del programa y las herramientas que utilizamos para implementarlo.
	\item \hyperlink{sec:gramatica}{Gramática, Parsing y AST}. Gramática utilizada, implementación de la misma y el árbol sintáctico resultante.
	\item \hyperlink{sec:referencias}{Ciclos y Referencias}. Métodos utilizados para detectar ciclos y resolver referencias entre tipos.
	\item \hyperlink{sec:json}{Generación de JSON}. Métodos utilizados para generar la salida en formato JSON.
\end{itemize}

\textbf{Palabras clave}: \textit{Parser, LALR(1), JSON Generator}.


\pagenumbering{arabic}

\hypertarget{sec:intro}{\section{Introducción}}
\label{sec:intro}

    El programa deberá ser capaz de leer definiciones de un subconjunto de tipos de Go. Estas definiciones consisten de asignarle un nombre a un tipo, donde el tipo puede ser:

\begin{itemize}
    \item \textbf{Tipo básico}. Representan los valores convencionales: \texttt{string}, \texttt{int}, \texttt{float64} y \texttt{bool}.
    \item \textbf{Arreglo}. Tienen la forma \texttt{[]}\textit{T} y representan listas de elementos de tipo \texttt{T}. Ejemplo: \texttt{[]float64}, \texttt{[][]int}, etc. 
    \item \textbf{Estructura}. Tienen la forma \texttt{struct \{ prop } \textit{T} \texttt{ ... \}}, donde \texttt{prop} es el nombre de una propiedad de la estructura y \texttt{T} el tipo de la misma. Las estructuras pueden tener más de una propiedad, cada una con su tipo asociado. Ejemplo:
    \begin{verbatim}
struct {
    nombre string
    pais struct {
        nombre string
        poblacion int
    }
}
    \end{verbatim}
    \item \textbf{Referencia}. Son nombres que hacen referencia a otro tipo definido en la entrada. Ejemplo:
    \begin{verbatim}
type persona {
    nombre cadena
    edad entero
}
type cadena string
type entero int
    \end{verbatim}
\end{itemize}

    Para hacer el parsing de estas estructuras, utilizaremos la biblioteca PLY \cite{PLY}. Esta consiste de dos módulos principales: Lex y Yacc. Lex se encarga de leer la entrada y convertirla en una lista de lexemas. Por otro lado, Yacc nos permitirá definir una gramática, utilizando estos lexemas como terminales, y generar un parser LALR(1) para ésta.
\hypertarget{sec:gramatica}{\section{Gramática, Parsing y AST}}


\subsection{Construcción de la Gramática}

    A continuación, analizaremos la siguiente cadena de entrada para determinar, de manera iterativa incremental, una gramática que genere el lenguaje pedido.

\begin{verbatim}
type persona struct {
    nombre string
    edad int
    nacionalidad pais
    ventas []float64
    activo bool
}
type pais struct {
    nombre string
    codigo struct {
        prefijo string
        sufijo string
    }
}
\end{verbatim}

    Podemos ver que, en la cadena de entrada, podemos tener varias declaraciones de tipos (\textit{DeclList}). Además, podemos asegurar que siempre vamos a tener al menos una (\textit{Decl}). Esta declaración es la que vamos a utilizar para generar el JSON de salida. Luego, agregamos las siguientes dos producciones.
\begin{equation}
\begin{split}
DeclList & \rightarrow Decl~|~Decl~DeclList 
\end{split}
\end{equation}

    Cada declaración está conformada de tres partes: la palabra reservada \textit{type}, un nombre (\textit{name}) y \textit{tipo} asociado al nombre.
\begin{equation} \label{gram:decl}
\begin{split}
Decl & \rightarrow type~name~Type
\end{split}
\end{equation}

    El no-terminal \texttt{Type} debe ser capaz de generar los tipos mencionados en la sección \hyperlink{sec:intro}{Introducción}. 
\begin{equation} \label{gram:type}
\begin{split}
Type & \rightarrow BasicType~|~ArrayType~|~StructType \\
BasicType & \rightarrow string~|~int~|~float64~|~bool \\
ArrayType & \rightarrow []~Type
\end{split}
\end{equation}

    Recordemos que los elementos de tipo estructura (\texttt{StructType}) pueden contener más de una propiedad, por lo que necesitaremos de otro no-terminal para listar las mismas.
\begin{equation} \label{gram:struct}
\begin{split}
StructType & \rightarrow struct~\{~PropList~\} \\
PropList & \rightarrow Prop~|~Prop~PropList \\
Prop & \rightarrow name~Type
\end{split}
\end{equation}

    En este punto, en busca de simplificar la gramática, podemos juntar los terminales \texttt{name}, \texttt{string}, \texttt{int}, \texttt{float64} y \texttt{bool} en un único terminal \texttt{id}. Este terminal va a representar a los \textit{identificadores}, tanto de propiedades como de tipos. Luego, en la etapa de parsing, diferenciaremos a cada identificador según el contexto y el valor que contenga el lexema. Reemplazando por \texttt{id} en \ref{gram:decl}, \ref{gram:type} y \ref{gram:struct} y juntando todo, obtenemos el siguiente conjunto de producciones.
\begin{equation}
\begin{split}
DeclList & \rightarrow Decl~|~Decl~DeclList  \\
Decl & \rightarrow type~id~Type \\
Type & \rightarrow BasicType~|~ArrayType~|~StructType \\
BasicType & \rightarrow id \\
ArrayType & \rightarrow []~Type \\
StructType & \rightarrow struct~\{~PropList~\} \\
PropList & \rightarrow Prop~|~Prop~PropList \\
Prop & \rightarrow id~Type
\end{split}
\end{equation}

    Notemos que aún podemos simplificar un poco más la gramática si reemplazamos los no-terminales del cuerpo de \texttt{Type} con las producciones correspondientes, obteniendo así $P$:

\begin{equation}
\begin{split}
DeclList & \rightarrow Decl~|~Decl~DeclList  \\
Decl & \rightarrow type~id~Type \\
Type & \rightarrow id~|~[]~Type~|~struct~\{~PropList~\} \\
PropList & \rightarrow Prop~|~Prop~PropList \\
Prop & \rightarrow id~Type
\end{split}
\end{equation}

    Con todo esto en cuenta, proponemos la siguiente gramática $G$ para generar las cadenas del lenguaje.
\[
G = \langle\{Decl, DeclList, Type, PropList, Prop\}, \{type, id, [], struct, '\{', '\}'\}, P, DeclList\rangle
\]

Dado que Yacc utiliza parsers LALR(1) para las gramáticas que recibe de entrada, es importante que la gramática \textit{G} sea LALR(1); en otras palabras, que durante el armado de la tabla para el parser no surja ningún conflicto. Ya que la misma librería se encarga de armar dicha tabla, nos informará si existe algún conflicto o no.

\subsection{Parser}

    Antes de implementar la gramática con Yacc, tenemos que hacer un pre-procesamiento para convertir el texto en una lista de tokens. Para esto, utilizaremos el módulo Lex y proveeremos las expresiones regulares necesarias para que el lexer pueda identificar a cada token.

    Definimos seis tokens en total, de los cuales dos son los literales '\{' y '\}'. El resto son:

\begin{itemize}
    \item \texttt{TYPE = type}
    \item \texttt{STRUCT = struct} 
    \item \texttt{ARRAY = []} 
    \item \texttt{ID = r'[a-z]\textbackslash w*'}
\end{itemize}

    Los primeros tres los definimos con la cadena que los identifica, mientras que el token \texttt{ID} lo representamos con una expresión regular que corresponde con los posibles identificadores de este lenguaje: son las palabras que empiecen con una letra minúscula y continúan con cualquier caracter (Unicode, incluidos '\_' y números \cite{re}).

    La implementación de la gramática es casi inmediata. Vamos a definir un método por cada producción (o por cada no-terminal si sus producciones son similares). 

\subsection{AST}
\label{subsec:ast}

    Para el armado del parser Yacc requiere que se defina una serie de funciones, donde cada una está asociada a una producción de la gramática y define el código a ejecutar una vez reconocida esa producción, siguiendo así el método conocido como \textit{traducción dirigida por sintaxis}.
    
    Con dichas funciones Yacc cuenta con lo necesario para armar la tabla para el parser LALR; para la gramática que propusimos, no reportó ningún conflicto, lo cual nos confirma que la gramática es efectivamente LALR(1).
    
    En cada método asociado a una producción de la gramática, vamos a armar progresivamente el árbol sintáctico (\textit{AST}). Para ello, vamos a declarar algunas estructuras que facilitan su construcción. 
    
    Para armar el árbol, asociaremos a cada no-terminal una estructura de datos particular:

\begin{itemize}
    \item \texttt{DeclList} y \texttt{PropList} se pueden implementar con los
        arreglos convencionales.
    \item \texttt{Decl}. Es una estructura \texttt{TypedefNode} con las propiedades
        \textit{nombre} (string) y \textit{tipo} (estructura de tipo).
    \item \texttt{Type}. Son las estructuras de tipo, y dependen de la
        producción utilizada:
    \begin{itemize}
        \item \texttt{ID}. 
        \begin{itemize}
            \item Si la cadena asociada al terminal es un tipo básico,
            lo representaremos con una estructura \texttt{BasicTypeNode} que
            contenga el nombre del tipo. 
            \item Caso contrario, lo representamos con la cadena original.
        \end{itemize} 
        Esta decisión será más evidente cuando hagamos la resolución de referencias en \hyperlink{sec:referencias}{Referencias}.
        \item \texttt{[] Type}. Es una estructura \texttt{ArrayTypeNode} con la
            propiedad \textit{tipo} (estructura de tipo).
        \item \texttt{Struct \{ PropList \}}: estructura
            \texttt{StructTypeNode} con una lista de propiedades \textit{Prop}'s.
    \end{itemize}
    \item \texttt{Prop}. Es una tupla $\langle \textit{nombre (string)}, \textit{tipo
        (estructura de tipo)}\rangle$.
\end{itemize}

    En el método de cada producción, instanciaremos la estructura asociada al no-terminal de la cabeza de la producción con las estructuras asociadas a los no-terminales del cuerpo de la producción. De esta forma, el resultado del parsing es el árbol dado por el arreglo de estructuras \texttt{TypedefNode}. 

\begin{itemize}
    \item Cada arreglo es un nodo que tiene como hijos a sus elementos.
    \item Cada estructura es otro nodo, que tene como hijo a la estructura que
        se encuentra en su propiedad \textit{tipo}.
    \item Las hojas son estructuras de tipos básicos, o bien cadenas que no
        son tipos básicos (referencias que se deben resolver).
\end{itemize}

\hypertarget{sec:referencias}{\section{Ciclos y Referencias}}


    Para completar el árbol y poder generar la salida, debemos antes resolver todas las referencias que aparezcan en el tipo principal (el primer elemento del arreglo que devuelve nuestro parser). Esto no siempre será posible, ya que en algunos casos estas referencias pueden generar \textbf{ciclos}.

\subsection{Detección de Ciclos}

    Para detectar estos ciclos, armamos el grafo dirigido de dependencias asociado al tipo principal, y lo recorremos marcando los nodos visitados hasta que nos encontremos en alguna de las siguientes situaciones:
\begin{itemize}
    \item O bien terminamos de recorrer todos, y no hay ciclos.
    \item O bien terminamos visitando un nodo que ya había sido previamente visitado, y hay un ciclo.
\end{itemize}

\begin{algorithm}[H]
\begin{algorithmic}
\Function{grafo\_dependencia}{$T$: TypedefNode[]}
    \State{$A \gets dict()$}
    \For {$t$ in $T$}
        \State {$A[t.nombre] \gets t.getDeps()$}
    \EndFor
    \State{\textbf{return} $A$}
\EndFunction
\end{algorithmic}
\caption{Construcción del grafo de dependencias}
\label{alg:grafo_dependencia}
\end{algorithm}

    En el algoritmo \ref{alg:grafo_dependencia} llamamos al método \texttt{getDeps} de TypedefNode, para obtener la lista de referencias. Éste, al mismo tiempo, vuelve a llamar a getDeps para sus hijos en el árbol sintáctico, propagando las cadenas encontradas nuevamente a la raíz.

\begin{algorithm}[H]
\begin{algorithmic}
    \Function{obtener\_ciclo}{$G$: grafo, $S$: string, $V$: diccionario}
    \State{\textbf{marcar} $S$ como visitado en $V$}
    \State{$siguientes \gets$ siguientes de $S$ en $G$}
    \For {$n$ in $siguientes$}
        \If{$n$ visitado en $V$}
            \State{\textbf{return} ref $\langle S, n \rangle$}
        \EndIf

        \State{$c \gets$ \texttt{obtener\_ciclo($G$, $S$, $V$)}}
        \If{$c$ no es None}
            \State{\textbf{return} $c$}
        \EndIf
    \EndFor
    \State{\textbf{return} None}
\EndFunction
\end{algorithmic}
\caption{Detección de ciclos}
\label{alg:obtener_ciclo}
\end{algorithm}

    En el algoritmo \ref{alg:obtener_ciclo}, además de detectar si el grafo tiene un ciclo o no, retornamos el arco del grafo que genera el ciclo para reportarlo como error.

    Es decir, si la detección de ciclos devuelve \texttt{None}, podemos asegurar que el tipo principal no tiene dependencias circulares y podemos continuar con la resolución de referencias. Notemos que el grafo solo tiene en cuenta las dependencias asociadas al tipo principal. Luego, si existen definiciones circulares que no involucran al tipo principal, el programa continuará sin reportar errores (siendo éste el comportamiento esperado). 

\subsection{Resolución de referencias}

    En la \autoref{subsec:ast}, mencionamos que aquellos identificadores que no sean tipos básicos, mantienen la cadena original. Para resolver estas referencias, utilizaremos el resultado del parser (la lista de \texttt{TypedefNode}'s) para crear un diccionario que tenga como claves los nombres de las declaraciones y el tipo como valor.

    Tanto la estructura \texttt{TypedefNode}, como las estructuras de los tipos tienen un método \texttt{resolve} que recibe como parámetro el diccionario anterior. Cada uno de estos intercambian las referencias que aparecen en sus propiedades por el valor correspondiente en el diccionario, y propagan la llamada a \texttt{resolve} en sus propiedades.

    Como verificamos que el tipo principal no tenía dependencias circulares, este proceso de resolver cada nodo del árbol eventualmente termina, y tenemos una árbol sintáctico completo que podemos usar para generar la salida.

\section{Resolución y Generación de JSON}
\label{sec:json}

\section{Conclusión}
\label{sec:conc}

    Podemos concluir que es posible implementar un programa para parsear definiciones de tipos (potencialmente cíclicas), utilizando TDS para la construcción del AST y métodos adicionales para la detección de ciclos. Además, pudimos generar cadenas JSON aleatorias que se corresponden con los tipos definidos dinámicamente.
    
    Además, podemos mencionar reflexiones más generales que aplican a cualquier desarrollo donde se tengan que procesar entradas pertenecientes a un cierto lenguaje (libre de contexto).
    
    En primer lugar, es necesario definir los tokens y, a partir de ellos, definir una gramática. En esta etapa pueden haber varias alternativas en cuanto a qué tan flexibles somos con las reglas aplicadas; por ejemplo, podemos tender a tener producciones específicas para las palabras reservadas, en nuestro caso los tipos básicos \texttt{int}, \texttt{string}, \texttt{bool} y \texttt{float64}, o tender a agrupar estas palabras en un único tipo de token \texttt{id} y luego considerar su valor en la etapa de traducción.
    
    Otra cosa a tener en cuenta es que hay muchas gramáticas distintas que general el mismo lenguaje. Algunas son menos intuitivas, más difíciles de leer o de procesar, más propensas a tener conflictos. En este trabajo, consideramos que la gramática propuesta es muy intuitiva, y aún así resultó ser LALR(1)-- esto siendo suficiente para que Yacc pueda armar el parser, sin tener que resolver conflictos de manera automatizada.
\printbibliography
\end{document}
