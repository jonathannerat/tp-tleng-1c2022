\phantomsection
\addcontentsline{toc}{section}{Resumen}


    En este trabajo implementaremos un programa para parsear definiciones de un subconjunto de tipos de Go \cite{Go}, y generar cadenas JSON \cite{JSON} aleatorias que se correspondan a los tipos definidos.

    El trabajo estará organizado de la siguiente manera:

\begin{itemize}
	\item \hyperlink{sec:intro}{Introducción}. Requerimientos del programa y las herramientas que utilizamos para implementarlo.
	\item \hyperlink{sec:gramatica}{Gramática, Parsing y AST}. Gramática utilizada, implementación de la misma y el árbol sintáctico resultante.
	\item \hyperlink{sec:referencias}{Ciclos y Referencias}. Métodos utilizados para detectar ciclos y resolver referencias entre tipos.
	\item \hyperlink{sec:json}{Generación de JSON}. Métodos utilizados para generar la salida en formato JSON.
\end{itemize}

\textbf{Palabras clave}: \textit{Parser, LALR(1), JSON Generator}.

