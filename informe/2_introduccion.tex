\section{Resumen}

En este trabajo implementaremos un programa para parsear definiciones de un subconjunto de tipos de Go, y generar cadenas JSON aleatorias que se correspondan a los tipos definidos.

El trabajo estará organizado de la siguiente manera:

\begin{itemize}
	\item \textbf{Sección \ref{sec:intro}: Introducción}, requerimientos del programa, y las herramientas que utilizamos para implementarlo
	\item \textbf{Sección \ref{sec:gramatica}: Gramática, Parser y AST}, gramática utilizada, implementación de la misma y el árbol sintáctico resultante.
	\item \textbf{Sección \ref{sec:referencias}: Ciclos y Referencias} métodos utilizados para detectar ciclos y resolver referencias entre tipos
	\item \textbf{Sección \ref{sec:json}: Generación de JSON} métodos utilizados para generar la salida aleatoria en formato JSON
\end{itemize}

Palabras clave: \textit{Parser, LALR(1), JSON Generator}

\newpage

\section{Introducción}
\label{sec:intro}

El programa deberá ser capaz de leer definiciones de un subconjunto de tipos de Go. Estas definiciones consisten de asignarle un nombre a un tipo, donde el tipo puede ser:

\begin{itemize}
    \item Tipo básico: \texttt{string}, \texttt{int}, \texttt{float64} y \texttt{bool}. Representan los valores convencionales.
    \item Arreglos: tienen la forma \texttt{[]}\textit{T} y representan listas de elementos de tipo \texttt{T}. Por ejemplo: \texttt{[]float64}, \texttt{[][]int}, etc. 
    \item Estructuras: tienen la forma \texttt{struct \{ prop } \textit{T} \texttt{ ... \}}, donde prop es el nombre de una propiedad de la estructura y T el tipo de la misma. Las estructuras tienen más de una propiedad, cada una con un tipo asociado. Por ejemplo:
    \begin{verbatim}
struct {
    nombre string
    pais struct {
        nombre string
        poblacion int
    }
}
    \end{verbatim}
    \item Referencias: son nombres que hacen referencia a otro tipo definido en la entrada. Por ejemplo:
    \begin{verbatim}
type persona {
    nombre cadena
    edad entero
}
type cadena string
type entero int
    \end{verbatim}
\end{itemize}

Para hacer el parsing de estas estructuras, utilizaremos la librería PLY (\textit{Python Lex-Yacc}). Esta consiste de dos módulos principales: Lex y Yacc. Lex es el que se encargará de leer la entrada y convertirla en una lista de lexemas, mientras que Yacc nos permitirá definir una gramática (utilizando estos lexemas como terminales) y generar el parser LALR(1) de la misma.
