
\hypertarget{sec:intro}{\section{Introducción}}
\label{sec:intro}

    El programa deberá ser capaz de leer definiciones de un subconjunto de tipos de Go. Estas definiciones consisten de asignarle un nombre a un tipo, donde el tipo puede ser:

\begin{itemize}
    \item \textbf{Tipo básico}. Representan los valores convencionales: \texttt{string}, \texttt{int}, \texttt{float64} y \texttt{bool}.
    \item \textbf{Arreglo}. Tienen la forma \texttt{[]}\textit{T} y representan listas de elementos de tipo \texttt{T}. Ejemplo: \texttt{[]float64}, \texttt{[][]int}, etc. 
    \item \textbf{Estructura}. Tienen la forma \texttt{struct \{ prop } \textit{T} \texttt{ ... \}}, donde \texttt{prop} es el nombre de una propiedad de la estructura y \texttt{T} el tipo de la misma. Las estructuras pueden tener más de una propiedad, cada una con su tipo asociado. Ejemplo:
    \begin{verbatim}
struct {
    nombre string
    pais struct {
        nombre string
        poblacion int
    }
}
    \end{verbatim}
    \item \textbf{Referencia}. Son nombres que hacen referencia a otro tipo definido en la entrada. Ejemplo:
    \begin{verbatim}
type persona {
    nombre cadena
    edad entero
}
type cadena string
type entero int
    \end{verbatim}
\end{itemize}

    Para hacer el parsing de estas estructuras, utilizaremos la biblioteca PLY \cite{PLY}. Esta consiste de dos módulos principales: Lex y Yacc. Lex se encarga de leer la entrada y convertirla en una lista de lexemas. Por otro lado, Yacc nos permitirá definir una gramática, utilizando estos lexemas como terminales, y generar un parser LALR(1) para ésta.