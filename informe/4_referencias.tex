\section{Ciclos y Referencias}
\label{sec:referencias}

Para completar el árbol y poder generar la salida, debemos antes resolver todas
las referencias que aparezcan en tipo principal (el primer elemento del arreglo
que devuelve nuestro parser). Esto no siempre será posible, ya que en algunos
casos estas referencias pueden generar ciclos.

\subsection{Detección de Ciclos}

Para detectar estos ciclos, armamos el grafo dirigido de dependencias asociado al tipo
principal, y lo recorremos marcando los nodos visitados hasta terminar de
recorrer todos (no hay ciclos), o hasta visitar un nodo previamente visitado
(hay un ciclo).

\begin{algorithm}[H]
\begin{algorithmic}
\Function{grafo\_dependencia}{$T$: TypedefNode[]}
    \State{$A \gets dict()$}
    \For {$t$ in $T$}
        \State {$A[t.nombre] \gets t.getDeps()$}
    \EndFor
    \State{\textbf{return} $A$}
\EndFunction
\end{algorithmic}
\caption{Construcción del grafo de dependencias}
\label{alg:grafo_dependencia}
\end{algorithm}

En el algoritmo \ref{alg:grafo_dependencia} llamamos al método \texttt{getDeps}
de TypedefNode, para obtener la lista de referencias. Este al mismo tiempo,
vuelve a llamar a getDeps para sus hijos en el arbol sintáctico, propagando las
cadenas encontrados nuevamente a la raíz.

\begin{algorithm}[H]
\begin{algorithmic}
    \Function{obtener\_ciclo}{$G$: grafo, $S$: string, $V$: diccionario}
    \State{\textbf{marcar} $S$ como visitado en $V$}
    \State{$siguientes \gets$ siguientes de $S$ en $G$}
    \For {$n$ in $siguientes$}
        \If{$n$ visitado en $V$}
            \State{\textbf{return}ref $\langle S, n \rangle$}
        \EndIf

        \State{$c \gets$ \texttt{obtener\_ciclo($G$, $S$, $V$)}}
        \If{$c$ no es None}
            \State{\textbf{return} $c$}
        \EndIf
    \EndFor
    \State{\textbf{return} None}
\EndFunction
\end{algorithmic}
\caption{Detección de ciclos}
\label{alg:obtener_ciclo}
\end{algorithm}

En el algoritmo \ref{alg:obtener_ciclo}, ademas de detectar si el grafo tiene un
ciclo o no, retornamos el arco del grafo que genera el ciclo para reportarlo
como error.

Notar que el grafo solamente tiene en cuenta las dependencias asociadas al tipo
principal, por lo que si en la entrada existen más definiciones de tipo que
generan dependencias circulares entre si, pero que no aparecen en el tipo
principal, el programa continuará sin reportar errores.

Por el contrario, si la detección de ciclos devuelve \texttt{None}, el tipo
principal no tiene dependencias circulares y podemos continuar con la resolución
de referencias.

\subsection{Resolución de referencias}

En la sección \ref{subsec:ast}, mencionamos que aquellos identificadores que no
sean tipos básicos, mantienen la cadena original. Para resolver estas
referencias, utilizaremos el resultado del parser (la lista de
\texttt{TypedefNode}'s) para crear un diccionario que tenga como claves los
las referencias, y como valor la estructura del tipo asociada.

Tanto la estructura \texttt{TypedefNode}, como las estructuras de tipos
tienen un método \texttt{resolve} que recibe como parámetro el diccionario
anterior. Cada uno de estos intercambian las referencias que aparecen en sus
propiedades por el valor correspondiente en el diccionario, y propagan la
llamada a \texttt{resolve} en sus propiedades.

Como verificamos que el tipo principal no tenía dependencias circulares, este
proceso de resolver cada nodo del árbol eventualmente termina, y tenemos un
arbol sintáctico completo, que podemos usar para generar la salida.

