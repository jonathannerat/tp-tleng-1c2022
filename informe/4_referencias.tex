\hypertarget{sec:referencias}{\section{Ciclos y Referencias}}


    Para completar el árbol y poder generar la salida, debemos antes resolver todas las referencias que aparezcan en el tipo principal (el primer elemento del arreglo que devuelve nuestro parser). Esto no siempre será posible, ya que en algunos casos estas referencias pueden generar \textbf{ciclos}.

\subsection{Detección de Ciclos}

    Para detectar estos ciclos, armamos el grafo dirigido de dependencias asociado al tipo principal, y lo recorremos marcando los nodos visitados hasta que nos encontremos en alguna de las siguientes situaciones:
\begin{itemize}
    \item O bien terminamos de recorrer todos, y no hay ciclos.
    \item O bien terminamos visitando un nodo que ya había sido previamente visitado, y hay un ciclo.
\end{itemize}

\begin{algorithm}[H]
\begin{algorithmic}
\Function{grafo\_dependencia}{$T$: TypedefNode[]}
    \State{$A \gets dict()$}
    \For {$t$ in $T$}
        \State {$A[t.nombre] \gets t.getDeps()$}
    \EndFor
    \State{\textbf{return} $A$}
\EndFunction
\end{algorithmic}
\caption{Construcción del grafo de dependencias}
\label{alg:grafo_dependencia}
\end{algorithm}

    En el algoritmo \ref{alg:grafo_dependencia} llamamos al método \texttt{getDeps} de TypedefNode, para obtener la lista de referencias. Éste, al mismo tiempo, vuelve a llamar a getDeps para sus hijos en el árbol sintáctico, propagando las cadenas encontradas nuevamente a la raíz.

\begin{algorithm}[H]
\begin{algorithmic}
    \Function{obtener\_ciclo}{$G$: grafo, $S$: string, $V$: diccionario}
    \State{\textbf{marcar} $S$ como visitado en $V$}
    \State{$siguientes \gets$ siguientes de $S$ en $G$}
    \For {$n$ in $siguientes$}
        \If{$n$ visitado en $V$}
            \State{\textbf{return} ref $\langle S, n \rangle$}
        \EndIf

        \State{$c \gets$ \texttt{obtener\_ciclo($G$, $S$, $V$)}}
        \If{$c$ no es None}
            \State{\textbf{return} $c$}
        \EndIf
    \EndFor
    \State{\textbf{return} None}
\EndFunction
\end{algorithmic}
\caption{Detección de ciclos}
\label{alg:obtener_ciclo}
\end{algorithm}

    En el algoritmo \ref{alg:obtener_ciclo}, además de detectar si el grafo tiene un ciclo o no, retornamos el arco del grafo que genera el ciclo para reportarlo como error.

    Es decir, si la detección de ciclos devuelve \texttt{None}, podemos asegurar que el tipo principal no tiene dependencias circulares y podemos continuar con la resolución de referencias. Notemos que el grafo solo tiene en cuenta las dependencias asociadas al tipo principal. Luego, si existen definiciones circulares que no involucran al tipo principal, el programa continuará sin reportar errores (siendo éste el comportamiento esperado). 

\subsection{Resolución de referencias}

    En la \autoref{subsec:ast}, mencionamos que aquellos identificadores que no sean tipos básicos, mantienen la cadena original. Para resolver estas referencias, utilizaremos el resultado del parser (la lista de \texttt{TypedefNode}'s) para crear un diccionario que tenga como claves los nombres de las declaraciones y el tipo como valor.

    Tanto la estructura \texttt{TypedefNode}, como las estructuras de los tipos tienen un método \texttt{resolve} que recibe como parámetro el diccionario anterior. Cada uno de estos intercambian las referencias que aparecen en sus propiedades por el valor correspondiente en el diccionario, y propagan la llamada a \texttt{resolve} en sus propiedades.

    Como verificamos que el tipo principal no tenía dependencias circulares, este proceso de resolver cada nodo del árbol eventualmente termina, y tenemos una árbol sintáctico completo que podemos usar para generar la salida.
