\section{Generación de JSON}
\label{sec:json}

Con el árbol completo, la generación del JSON aleatorio se puede describir de
manera recursiva, explicando como se generan los casos base (tipos básicos), y
luego el paso inductivo (tipos que tienen uno o más tipos en sus propiedades):

\paragraph{Casos base}

\begin{itemize}
    \item \texttt{int}, \texttt{float64} y \texttt{bool} son triviales con la
        librería \texttt{random} de Python.
    \item \texttt{string}: utilizamos la cadena ``abcdefghijklmnñopqrstuvwxyz''
        como diccionario de caractéres, y tomamos una cantidad aleatoria de
        caracteres (máximo 20) de manera aleatoria (nuevamente con
        \texttt{random}),
\end{itemize}

\paragraph{Paso inductivo}

\begin{itemize}
    \item \texttt{[] <tipo>}: generamos la representación en JSON de \texttt{<tipo>} una
        cantidad aleatoria de veces (entre 1 y 5), y los listamos separados por comas, y
        encerrados entre corchetes:\\
        \texttt{[<tipo 1>,<tipo 2>,\dots]}
    \item \texttt{struct \{ <prop 1>~<tipo 1>~... \}}: por cada propiedad del
        scruct, generamos la representación en JSON del tipo correspondiente, y
        los representamos como un objeto convencional: \\
        \texttt{\{"prop 1":<tipo 1>, \dots\}}
\end{itemize}

\subsection{Ejemplos}

A continuación mostramos algunos ejemplos de entradas válidas, inválidas, y los
errores que se reportan para las inválidas, separando los resultados por una
linea con guiones:

\begin{lstlisting}[caption=Ejemplo válido]
type persona struct {
    nombre  string
    edad    int
    altura  float64
    activo  bool
}
----
{
  "nombre": "nvjvdkibhukontfzdco",
  "edad": 456,
  "altura": 56.26,
  "activo": true,
}
\end{lstlisting}

\newpage

\begin{lstlisting}[caption=Ejemplo inválido: ciclos]
type persona struct {
    nombre  string
    edad    int
    nacionalidad pais
}

type pais struct {
    nombre string
    presidente persona
}
----
Error: se encontró una referencia circular en la definición de 'pais', mediante el tipo 'persona'
\end{lstlisting}

\begin{lstlisting}[caption=Ejemplo inválida: no definido]
type persona struct {
    nombre  string
    edad    int
    nacionalidad pais
}
----
Error: el tipo 'pais' no esta definido.
\end{lstlisting}
