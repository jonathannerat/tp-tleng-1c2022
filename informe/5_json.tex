\hypertarget{sec:json}{\section{Generación de JSON}}

    Con el árbol completo, la generación del JSON aleatorio se puede describir de manera recursiva. Esto es, cómo se generan los casos base (tipos básicos), y cómo se genera el caso recursivo (tipos que tienen uno o más tipos en sus propiedades):

\paragraph{Casos base}

\begin{itemize}
    \item \texttt{int}, \texttt{float64} y \texttt{bool} son triviales con la
        biblioteca \texttt{random} de Python.
    \item \texttt{string}: utilizamos la cadena ``abcdefghijklmnñopqrstuvwxyz''
        como diccionario de caracteres, y tomamos al azar una cantidad aleatoria de
        caracteres (máximo 20), nuevamente mediante \texttt{random},
\end{itemize}

\paragraph{Caso Recursivo}

\begin{itemize}
    \item \texttt{[] <tipo>}: generamos la representación en JSON de \texttt{<tipo>} una
        cantidad aleatoria de veces (entre 1 y 5), y los listamos separados por comas y
        encerrados entre corchetes:\\
        \texttt{[<tipo 1>,<tipo 2>,\dots]}
    \item \texttt{struct \{ <prop 1>~<tipo 1>~... \}}: por cada propiedad del
        struct, generamos la representación en JSON del tipo correspondiente, y
        los representamos como un objeto convencional: \\
        \texttt{\{"prop 1":<tipo 1>, \dots\}}
\end{itemize}

    Este algoritmo recursivo toma como punto de partida el tipo asociado a la primera definición encontrada; en otras palabras, al tipo del primer \texttt{TypedefNode} del arreglo computado por el parser.

\subsection{Ejemplos}

    A continuación, mostramos algunos ejemplos de entradas válidas, inválidas y los errores que se reportan para las inválidas.

\subsubsection{Ejemplos válidos}

Los ejemplos que siguen prueban que las salidas de JSON aleatorizadas sean las esperadas dada una serie de definiciones válida.

\begin{lstlisting}[caption=Ejemplo con un \texttt{struct}.]
type persona struct {
    nombre  string
    edad    int
    altura  float64
    activo  bool
}
---- RESULTADO
{
  "nombre": "nvjvdkibhukontfzdco",
  "edad": 456,
  "altura": 56.26,
  "activo": true,
}
\end{lstlisting}

\newpage

\begin{lstlisting}[caption={Ejemplo complejo con referencias, arreglos y structs anidados.}]
type user struct {
    name string
    email string
    the matrix
    projects []project
}

type project struct {
    tasks []task
    created_at struct {
        day int
        month int
        year int
    }
}

type task struct {
    description string
    done bool
}

type complex struct {
    real float64
    imag float64
}

type matrix [][]complex

---- RESULTADO

{
  "name": "nufaffnbsf",
  "email": "cr",
  "the": [
    [
      {"real":372.36, "imag": 47.56},
      {"real":209.7, "imag": 581.6}
    ],
    [
      {"real": 408.72, "imag": 247.98},
      {"real": 784.06, "imag": 302.47}
    ]
  ],
  "projects": [
    {
      "tasks": [
        {
          "description": "fpnudndruzpxm",
          "done": true,
        },
        {
          "description": "yerqsjzm",
          "done": false,
        }
      ],
      "created_at": {
        "day": 424,
        "month": 694,
        "year": 762
      }
    },
  ]
}
\end{lstlisting}


\subsubsection{Ejemplos con errores semánticos}

    Los siguientes ejemplos prueban la detección de ciclos y tipos indefinidos desarrollada.
    Además se incluyen otros ejemplos de redefinición de tipos existentes (básicos o referencias).

\begin{lstlisting}[caption=Ejemplo con ciclo de definiciones que parte del tipo principal.]
type persona struct {
    nombre  string
    edad    int
    nacionalidad pais
}

type pais struct {
    nombre string
    presidente persona
}
---- RESULTADO

Error: se encontró una referencia circular en la definición de 'pais', mediante el tipo 'persona'
\end{lstlisting}

\begin{lstlisting}[caption=Ejemplo con uso de tipo no definido.]
type persona struct {
    nombre  string
    edad    int
    nacionalidad pais
}
---- RESULTADO

Error: el tipo 'pais' no esta definido.
\end{lstlisting}

\begin{lstlisting}[caption=Ejemplo con redefinición de tipo básico.]
type persona struct {
    nombre string
}

type string bool
---- RESULTADO

Error: nombre de tipo inválido, 'string' representa un tipo básico.
\end{lstlisting}

\begin{lstlisting}[caption=Ejemplo con múltiples declaraciones para el mismo tipo.]
type persona struct {
    nombre string
    nacimiento fecha
}

type fecha struct {
    dia int
    mes int
    año int
}

type fecha string
---- RESULTADO

Error:  múltiples declaraciones del tipo 'fecha'
\end{lstlisting}

\subsubsection{Ejemplos con errores sintácticos}

En los ejemplos que siguen probamos la detección de errores sintácticos de acuerdo a las reglas correspondientes al lexer y al parser que definimos con Lex y Yacc respectivamente.

\begin{lstlisting}[caption=Ejemplo con caracter ilegal no reconocido por el lexer.]
type Persona struct {
    nombre string
    edad int
}
---- RESULTADO

Error: caractér ilegal 'P' en la linea 1, columna 6
\end{lstlisting}

\begin{lstlisting}[caption=Ejemplo donde se usa la palabra reservada \texttt{struct} después de \texttt{type} cuando se esperaba un identificador.]
type struct struct {
    nombre string
    edad int
}
---- RESULTADO

Error: token inválido 'struct' en la linea 1, columna 6
\end{lstlisting}